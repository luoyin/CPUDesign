\documentclass[10pt]{book}
\usepackage{amsmath}
\usepackage{amssymb}
\usepackage{latexsym}
\usepackage{ctex}
\usepackage{mathdots}
\usepackage{mathrsfs}
\usepackage{longtable}
\usepackage{supertabular}
\usepackage{multirow}
\usepackage{multicol}
\usepackage{array}
\usepackage{color, framed}
\usepackage{xcolor}
\usepackage{float}
\usepackage{listings}
\usepackage{threeparttable}
\usepackage{graphicx}
\usepackage{graphics}
\usepackage{enumerate}
\usepackage{footmisc}
\usepackage{comment}
\usepackage{tikz}
\usetikzlibrary{shapes.geometric}
\usepackage{lscape}
\usepackage[colorlinks]{hyperref}
\usepackage{geometry}
\setcounter{secnumdepth}{4}
\geometry{left=2.0cm, right=2.0cm, top=2.0cm, bottom=2.0cm}

\begin{document}

\lstset{numbers=left, 
	numberstyle= \tiny, 
	keywordstyle= \color{ blue!70},
	commentstyle=\color{red!50!green!50!blue!50},
	frame=shadowbox, 
	rulesepcolor= \color{ red!20!green!20!blue!20},
	xleftmargin=-1em,
	xrightmargin=-1em,
	tabsize=4,
	breaklines=true,
    basicstyle=\small
}

\definecolor{shadecolor}{rgb}{0.92, 0.92, 0.92}

\newcommand{\red}[1]{\textcolor[rgb]{1.0, 0.0, 0.0}{#1}}
\newcommand{\green}[1]{\textcolor[rgb]{0.0, 1.0, 0.0}{#1}}
\newcommand{\blue}[1]{\textcolor[rgb]{0.0, 0.0, 1.0}{#1}}
\newcommand{\mypath}[1]{/home/luoyin/Notes2016/ECG_Analysis/{#1}/}

\title{CPU设计: 8008}
\author{luoyin}
\date{}
\maketitle

\tableofcontents

\chapter{微程序设计}
\section{子程序设计}
\begin{itemize}
  \item IF (Instruction Fetch): T1-T2-T3 (PCI)
  \item MR (Memory Read): T1-T2-T3 (PCR)
  \item MW (Memory Write): T1-T2-T3 (PCW)
  \item RR (Register Read): T4
  \item RW (Register Write): T5
  \item PCU (PC Update): T4-T5
  \item IOR (I/O Read): T3-T4-T5
\end{itemize}

\section{指令组成}
\begin{landscape}
\begin{longtable}{|l|l|l|}
\caption{指令组成} \label{tb_ins_comb}\\
\hline
指令 & 指令码 & 组成 \\
\hline
\endfirsthead
\caption{指令组成(续)} \\
\hline
指令 & 指令码 & 组成 \\
\hline
\endhead
\hline
\endfoot
Lrr 			& 11DDDSSS & PCO(PCL-PCH)-IF-rR-rW \\
LrM				& 11DDD111 & PCO(PCL-PCH)-IF-MA(rLO-rMO)-MR-X1-rW \\
LMr				& 11111SSS & PCO(PCL-PCH)-IF-rR-MA(rLO-rMO)-MW \\
LrI				& 00DDD110 & PCO(PCL-PCH)-IF-PCO(PCL-PCH)-IMMb-X1-rW \\
\hline
INr/DCr			& 00DDD00V & PCO(PCL-PCH)-IF-X1-rW \\
\hline
ALU OP r		& 10PPPSSS & PCO(PCL-PCH)-IF-rR-rW \\
ALU OP M 		& 10PPP111 & PCO(PCL-PCH)-IF-MA(rLO-rMO)-MR-X1-rW \\
ALU OP I 		& 00PPP100 & PCO(PCL-PCH)-IF-PCO(PCL-PCH)-IMMb-X1-rW \\
ROT 			& 000VV010 & PCO(PCL-PCH)-IF-X1-rW \\
\hline
JMP				& 01XXX100 & PCO(PCL-PCH)-IF-PCO(PCL-PCH)-IMMb-PCO(PCL-PCH)-IMMa1-PCU(PCHU-PCLU) \\
JFc/JTc			& 01VCC000 & PCO(PCL-PCH)-IF-PCO(PCL-PCH)-IMMb-PCO(PCL-PCH)-IMMa2-PCU(PCHU-PCLU) \\
CAL				& 01XXX110 & PCO(PCL-PCH)-IF-PCO(PCL-PCH)-IMMb-PCO(PCL-PCH)-IMMa3-PCU(PCHU-PCLU) \\
CFc/CTc			& 01VCC010 & PCO(PCL-PCH)-IF-PCO(PCL-PCH)-IMMb-PCO(PCL-PCH)-IMMa4-PCU(PCHU-PCLU) \\
RET				& 00XXX111 & PCO(PCL-PCH)-IF-POP-X2 \\
RFc/RTc			& 00VCC011 & PCO(PCL-PCH)-IF-POPc(c)-X2 \\
\hline
INP				& 0100MMM1 & PCO(PCL-PCH)-IF-IO(rAO-rBO)-IOb-CO-rW \\
OUT				& 01RRMMM1 & PCO(PCL-PCH)-IF-IO(rAO-rBO)-X0 \\
\hline
HLT				& 0000000X & PCO(PCL-PCH)-IF \\
HLT				& 11111111 & PCO(PCL-PCH)-IF \\
\end{longtable}
\end{landscape}

\section{指令执行状态变化}

\section{微指令设计}
\subsection{微指令组成}
\begin{itemize}
  \item 状态码 (3位)
  \item 寄存器组操作 (2位): 输出使能, 写使能
  \item 
\end{itemize}

\subsection{指令分类}
\begin{itemize}
  \item $D_{7}D_{6}=00$: 特殊指令
  \begin{itemize}
    \item $D_{2}D_{1}D_{0}=000$: HLT
    \item $D_{2}D_{1}D_{0}=001$: HLT
    \item $D_{2}D_{1}D_{0}=010$: ROT
    \item $D_{2}D_{1}D_{0}=011$: RFc/RTc
    \item $D_{2}D_{1}D_{0}=100$: ALU OP I
    \item $D_{2}D_{1}D_{0}=101$: RST
    \item $D_{2}D_{1}D_{0}=110$: LrM/LrI
    \item $D_{2}D_{1}D_{0}=111$: RET
  \end{itemize}
  \item $D_{7}D_{6}=01$: 跳转指令
  \begin{itemize}
    \item $D_{2}D_{1}D_{0}=000$: JFc/JTc
    \item $D_{2}D_{1}D_{0}=010$: CFc/CTc
    \item $D_{2}D_{1}D_{0}=100$: JMP
    \item $D_{2}D_{1}D_{0}=110$: CAL
    \item $D_{2}D_{1}D_{0}=XX1$: INP/OUT
  \end{itemize}
  \item $D_{7}D_{6}=10$: 算术指令
  \item $D_{7}D_{6}=11$: 寄存器指令
\end{itemize}

\subsection{微程序分类与跳转}
使用$D_{7}D_{6}$进行一次分组, 使用$D_{2}D_{1}D_{0}$进行二次分组

\subsection{微指令转移}
微指令转移按照如下计算规则:
\begin{equation}
A_{i}=\mu A_{i}+\sum P^{I}_{i}I_{i}+\sum P^{S}_{i}S_{i}+\sum P^{C}_{i}C_{i}
\end{equation}
其中, $\mu A_{i}$为微指令中的下一指令段, $P$为微指令中的控制段, 按作用类型不同分为指令控制段$P^{I}_{i}$, 状态控制段$P^{S}_{i}$, 和条件控制段$P^{C}_{i}$, $I_{i}$为指令寄存器的位段, $S_{i}$为状态寄存器的位段, $C_{i}$为条件判定寄存器的位段.

\subsection{微指令转移方式}
\begin{itemize}
  \item 直接转移: 微指令中的控制段均为0, 微指令运行下一指令直接由微指令中的$\mu A_{i}$段决定.
  \item 按指令转移: 微指令中的指令控制段$P^{I}_{i}$不为0, 此时, 微指令中的$\mu A_{i}$段决定跳转时的基址, $\sum P^{I}_{i}I_{i}$决定偏移量.
  \item 按状态转移: 微指令中的状态控制段$P^{S}_{i}$不为0, 此时, 微指令中的$\mu A_{i}$段决定跳转时的基址, $\sum P^{S}_{i}S_{i}$决定偏移量.
  \item 按条件转移: 微指令中的条件控制段$P^{C}_{i}$不为0, 此时, 微指令中的$\mu A_{i}$段决定跳转时的基址, $\sum P^{C}_{i}C_{i}$决定偏移量.
  \item 复合转移: 微指令中的$\mu A_{i}$段决定跳转时的基址, 结合指令控制段$P^{I}_{i}$, 状态控制段$P^{S}_{i}$, 和条件控制段$P^{C}_{i}$综合决定偏移量.
\end{itemize}

\subsection{微指令转移方式}
\begin{landscape}
\begin{longtable}{|l|l|l|l|l|l|l|}
\caption{微程序表} \label{tb_micro_prog} \\
\hline
地址 & 微指令 & 状态 & 功能 & 下一微指令 & 下一状态 & 转移类型             \\
\hline
\endfirsthead
\caption{模块列表(续)} \\
\hline
地址 & 微指令 & 状态 & 功能 & 下一微指令 & 下一状态 & 转移类型             \\
\hline
\endhead
\hline
\endfoot
0	& PCL	& T1 	& PCL输出 				& PCH 								& T2 			& 直接转移 \\
1	& PCH 	& T2 	& PCH输出 				& IF, IMMa, IMMb					& T3, WAIT 		& 状态转移 \\
2	& IF  	& T3 	& DATA to IR and regB 	& rR, rLO, PCL, POP, POPc, rAO, X1	& T4, T1, HLT	& 指令转移, 状态转移, 条件转移 \\ 
& rR  	& T4 	& reg Read  			& rW, rLO		 					& T5, T1, HLT 	& 指令转移, 状态转移  \\
& rW  	& T5 	& reg Write 			& PCL								& T1, INT		& 指令转移, 状态转移 \\
& rLO 	& T1 	& reg L Out 			& rHO								& T2			& 直接转移 \\
& rHO 	& T2 	& reg H Out 			& MR, MW							& T3, WAIT		& 指令转移, 状态转移 \\
& MR  	& T3	& Memory Read 			& X1								& T4			& 直接转移 \\
& MW  	& T3	& Memory Write 			& PCL								& T1, INT		& 状态转移 \\ 
\end{longtable}
\end{landscape}

\subsection{跳转设计}
\subsubsection{IF跳出}
跳出指向
\begin{itemize}
  \item rR: Lrr+LMr (11VVVSSS), ALU op r (10PPPSSS), 合并(1XXXXSSS, SSS<>111)
  \item rLO: LrM (11DDD111, DDD<>111), ALU op M (10PPP111)
  \item rAO: INP+OUT (01XXXXX1)
  \item POP: RETURN (00XXXX11)
  \item PCL: JUMP, CALL (01XXXXX0)
  \item PCL(next): HLT, INT, NORMAL
  \item X1: INr/DCr
\end{itemize}

\subsubsection{指令前缀00}
指令通过$D_{2}D_{1}D_{0}$进行分类
\begin{itemize}
  \item 000: HLT/INr (通过$D_{5}D_{4}D_{3}$进行分类). IF-STOP/IF-X1
  \item 001: HLT/DCr (通过$D_{5}D_{4}D_{3}$进行分类). IF-STOP/IF-X1
  \item 010: ROT (RLC, RRC, RAL, RAR, 通过$D_{5}D_{4}D_{3}$进行分类). IF-X1
  \item 011: RFc/RTc. IF-POP
  \item 100: ALU op I. IF-PCL
  \item 101: RST. IF-PCLU2-PCHU2
  \item 110: LrI/LMI (通过$D_{5}D_{4}D_{3}$进行分类). IF-PCL
  \item 111: RET. IF-POP
\end{itemize}

将微程序地址10000-10111与上面8个指令对应. 指令跳转表达式为
\begin{eqnarray}
J_{C} &=& \bar D_{7}\bar D_{6}(\overline{\bar D_{2}D_{1}D{0}}+\bar D_{2}D_{1}D_{0}J) \\
A_{4} &=& J_{C}1 \\
A_{3} &=& J_{C}P_{0}(D_{7}+D_{6}) \\
A_{2} &=& J_{C}(\mu A_{2}+P_{0}\bar D_{7}\bar D_{6}I_{2}) \\
A_{1} &=& J_{C}(\mu A_{1}+P_{0}\bar D_{7}\bar D_{6}I_{1}) \\
A_{0} &=& J_{C}(\mu A_{0}+P_{0}\bar D_{7}\bar D_{6}I_{0}) 
\end{eqnarray}

\subsubsection{指令前缀01}
指令通过$D_{2}D_{1}D_{0}$进行分类
\begin{itemize}
  \item 000: JFc/JTc
  \item 010: CFc/CTc
  \item 100: JMP
  \item 110: CAL
  \item XX1: INP/OUT
\end{itemize}

将微程序地址11000-11111与上面8个指令对应. 指令跳转表达式为
\begin{eqnarray}
J_{C} &=& \bar D_{7}D_{6}(\overline{\bar D_{2}\bar D_{0}}+\bar D_{2}\bar D_{0}J) \\
A_{4} &=& \bar J_{C}1 \\
A_{3} &=& \bar J_{C}(P_{0}\bar D_{7}D_{6}) \\
A_{2} &=& \bar J_{C}(\mu A_{2}+P_{1}\bar D_{7}D_{6}I_{0}) \\
A_{1} &=& \bar J_{C}(\mu A_{2}+P_{1}\bar D_{7}D_{6}I_{2}) \\
A_{0} &=& \bar J_{C}(\mu A_{2}+P_{1}\bar D_{7}D_{6}I_{1}\bar I_{0}+P_{1}\bar D_{7}D_{6}\bar I_{5}\bar I_{4}I_{0})
\end{eqnarray}

\subsubsection{指令前缀10}
指令通过$D_{2}D_{1}D_{0}$进行分类
\begin{itemize}
  \item SSS: ALU op r
  \item 111: ALU op M
\end{itemize}

将微程序地址100000-100001与上面2个指令对应. 指令跳转表达式为
\begin{eqnarray}
A_{5} &=& D_{7}\bar D_{6} \\
A_{4} &=& \overline{D_{7}\bar D_{6}} \\
A_{3} &=& \overline{D_{7}\bar D_{6}} \\
A_{2} &=& \overline{D_{7}\bar D_{6}} \\
A_{1} &=& \overline{D_{7}\bar D_{6}} \\
A_{0} &=& D_{7}\bar D_{6}I_{2}I_{1}I_{0}
\end{eqnarray}

\subsubsection{指令前缀11}
指令通过$D_{5}D_{4}D_{3}D_{2}D_{1}D_{0}$进行分类
\begin{itemize}
  \item DDDSSS: Lrr
  \item DDD111: LrM
  \item 111SSS: LMr
  \item 111111: HLT
\end{itemize}

将微程序地址100100-100111与上面4个指令对应. 指令跳转表达式为
\begin{eqnarray}
A_{5} &=& D_{7}D_{6} \\
A_{4} &=& \overline{D_{7}D_{6}} \\
A_{3} &=& \overline{D_{7}D_{6}} \\
A_{2} &=& D_{7}D_{6} \\
A_{1} &=& D_{7}D_{6}I_{5}I_{4}I_{3} \\
A_{0} &=& D_{7}D_{6}I_{2}I_{1}I_{0}
\end{eqnarray}

\subsubsection{条件跳转}
适用指令: JFc/JTc, CFc/CTc, RFc/RTc, 引入条件判定变量$J$, 当条件成立时$J=1$, 否则$J=0$, 指令跳转表达式为
\begin{equation}
A_{i}=J()
\end{equation}

\subsection{微指令组合逻辑}
\begin{itemize}
  \item srcM: $D_{2}D_{1}D_{0}$
  \item dstM: $D_{5}D_{4}D_{3}$
  \item JUMP: 
\end{itemize}

\subsection{微指令表}
\begin{longtable}{|l|l|l|c|c|c|c|c|c|c|c|c|c|c|}
\caption{微指令表} \label{tb_micro_prog} \\
\hline
\multirow{2}{1cm}{地址} & \multirow{2}{1.5cm}{指令} & \multirow{2}{1.5cm}{微指令} & \multicolumn{3}{c|}{S} & \multicolumn{3}{c|}{P} & \multicolumn {5}{c|}{$\mu A$} \\
\cline{4-14}
                       &             &             & 2 & 1 & 0 & 2 & 1 & 0 & 4 & 3 & 2 & 1 & 0 \\
% 地址 & 微指令 & S2 & S1 & S0 & $P_{0}$ & $\mu A_{2}$ & $\mu A_{1}$ & $\mu A_{0}$             \\
\hline
\endfirsthead
\caption{微指令表(续)} \\
\hline
\multirow{2}{1cm}{地址} & \multirow{2}{1.5cm}{指令} & \multirow{2}{1.5cm}{微指令} & \multicolumn{3}{c|}{S} & \multicolumn{3}{c|}{P} & \multicolumn {5}{c|}{$\mu A$} \\
\cline{4-14}
                       &             &             & 2 & 1 & 0 & 2 & 1 & 0 & 4 & 3 & 2 & 1 & 0 \\
\hline
\endhead
\hline
\endfoot
000000	&  			& PCL		& 0 & 1 & 0   & 0 & 0 & 0   & 0 & 0 & 0 & 0 & 1 \\
000001 	& 			& PCH		& 1 & 0 & 0   & 0 & 0 & 0   & x & x & x & x & x \\
000010 	& 			& IF   		& 0 & 0 & 1   & x & x & x   & 0 & 1 & x & x & x \\
001000 	& 			& rR   		& 1 & 1 & 1   & x & x & x   & x & x & x & x & x \\
001001 	& 			& POP  		& 1 & 1 & 1   & 0 & 0 & 0   & x & x & x & x & x \\
001010 	& 			& X1   		& 1 & 1 & 1   & 0 & 0 & 0   & x & x & x & x & x \\
001100 	& 			& rLO		& 0 & 1 & 0   & 0 & 0 & 0	& x & x & x & x & x \\
001101 	&			& rAO  		& 0 & 1 & 0   & 0 & 0 & 0   & x & x & x & x & x \\
001110 	&			& PCL2 		& 0 & 1 & 0   & 0 & 0 & 0   & x & x & x & x & x \\
010000	& INr		& 			& 1 & 1 & 1	  &   &   &     & x & x & x & x & x \\
010001	& DCr		& 			& 1 & 1 & 1	  &   &   &     & x & x & x & x & x \\
010010	& ROT		&			& 1 & 1 & 1	  &   &   &     & x & x & x & x & x \\
010011	& RETc		&			& 1 & 1 & 1	  &   &   &     & x & x & x & x & x \\
010100	& ALU op I 	& 			& 1 & 1 & 1	  &   &   &     & x & x & x & x & x \\
010101	& RST		&			& 1 & 1 & 1	  &   &   &     & x & x & x & x & x \\
010110	& LrI/LMI	&			& 1 & 1 & 1	  &   &   &     & x & x & x & x & x \\
010111	& RET		&			& 1 & 1 & 1	  &   &   &     & x & x & x & x & x \\
011000	& JMPc		&			& 1 & 1 & 1	  &   &   &     & x & x & x & x & x \\
011001	& CALc		&			& 1 & 1 & 1	  &   &   &     & x & x & x & x & x \\
011010	& JMP		&			& 1 & 1 & 1	  &   &   &     & x & x & x & x & x \\
011011	& CAL		&			& 1 & 1 & 1	  &   &   &     & x & x & x & x & x \\
011100	& INP		&			& 1 & 1 & 1	  &   &   &     & x & x & x & x & x \\
011101	& OUT		&			& 1 & 1 & 1	  &   &   &     & x & x & x & x & x \\
100000	& ALU op r 	&			& 1 & 1 & 1	  &   &   &     & x & x & x & x & x \\
100001	& ALU op M 	&			& 1 & 1 & 1	  &   &   &     & x & x & x & x & x \\
100100	& Lrr		&			& 1 & 1 & 1	  &   &   &     & x & x & x & x & x \\
100101	& LrM		&			& 1 & 1 & 1	  &   &   &     & x & x & x & x & x \\
100110	& LMr		&			& 1 & 1 & 1	  &   &   &     & x & x & x & x & x \\
100111	& HLT		&			& 1 & 1 & 1	  &   &   &     & x & x & x & x & x \\
\end{longtable}

\section{微指令设计}
\subsection{指令译码}
\begin{eqnarray}
Lrr 	&=& D_{7}D_{6}\overline{D_{5}D_{4}D_{3}}~\overline{D_{2}D_{1}D_{0}} \\
LrM 	&=& D_{7}D_{6}\overline{D_{5}D_{4}D_{3}}D_{2}D_{1}D_{0} \\
LMr 	&=& D_{7}D_{6}D_{5}D_{4}D_{3}\overline{D_{2}D_{1}D_{0}} \\
LrI 	&=& \bar{D_{7}}\bar{D_{6}}\overline{D_{5}D_{4}D_{3}}D_{2}D_{1}\bar{D_{0}} \\
LMI 	&=& \bar{D_{7}}\bar{D_{6}}D_{5}D_{4}D_{3}D_{2}D_{1}\bar{D_{0}} \\
INr 	&=& \bar{D_{7}}\bar{D_{6}}\overline{D_{5}D_{4}D_{3}}\bar{D_{2}}\bar{D_{1}}\bar{D_{0}} \\
DCr 	&=& \bar{D_{7}}\bar{D_{6}}\overline{D_{5}D_{4}D_{3}}\bar{D_{2}}\bar{D_{1}}D_{0} \\
ALUopR 	&=& D_{7}\bar{D_{6}}\overline{D_{2}D_{1}D_{0}} \\
ALUopM 	&=& D_{7}\bar{D_{6}}D_{2}D_{1}D_{0} \\
ALUopI 	&=& \bar{D_{7}}\bar{D_{6}}D_{2}\bar{D_{1}}\bar{D_{0}} \\
ROT		&=& \bar{D_{7}}\bar{D_{6}}\bar{D_{2}}D_{1}\bar{D_{0}} \\
JMP		&=& \bar{D_{7}}D_{6}D_{2}\bar{D_{1}}\bar{D_{0}} \\
JMPc	&=& \bar{D_{7}}D_{6}\bar{D_{2}}\bar{D_{1}}\bar{D_{0}} \\
CAL		&=& \bar{D_{7}}D_{6}D_{2}D_{1}\bar{D_{0}} \\
CALc	&=& \bar{D_{7}}D_{6}\bar{D_{2}}D_{1}\bar{D_{0}} \\
RET		&=& \bar{D_{7}}\bar{D_{6}}D_{2}D_{1}D_{0} \\
RETc	&=& \bar{D_{7}}\bar{D_{6}}\bar{D_{2}}D_{1}D_{0} \\
RST		&=& \bar{D_{7}}\bar{D_{6}}D_{2}\bar{D_{1}}D_{0} \\
INP		&=& \bar{D_{7}}D_{6}\bar{D_{5}}\bar{D_{4}}D_{0} \\
OUT		&=& \bar{D_{7}}D_{6}(\bar{D_{5}}+\bar{D_{4}})D_{0} \\
HLT		&=& \bar{D_{7}}\bar{D_{6}}\bar{D_{5}}\bar{D_{4}}\bar{D_{3}}\bar{D_{2}}\bar{D_{1}}+D_{7}D_{6}D_{5}D_{4}D_{3}D_{2}D_{1}D_{0}
\end{eqnarray}
将指令分段
\begin{eqnarray}
M_s		&=& D_{2}D_{1}D_{0} \\
M_d		&=& D_{5}D_{4}D_{3}  
\end{eqnarray}
指令编码公式如下
\begin{eqnarray}
I_{Lrr}		&=& D_7 D_6\overline{M_d}~\overline{M_s} \\
I_{LrM}		&=& D_7 D_6\overline{M_d}M_s \\
I_{LMr}		&=& D_7 D_6 M_d\overline{M_s} \\
I_{LrI}		&=& \bar{D_7}\bar{D_6}\overline{M_d}D_2 D_1\bar{D_0} \\
I_{LMI}		&=& \bar{D_7}\bar{D_6}M_d D_2 D_1\bar{D_0} \\
I_{INr}		&=& \bar{D_7}\bar{D_6}\overline{M_d}\bar{D_2}\bar{D_1}\bar{D_0} \\
I_{DCr}		&=& \bar{D_7}\bar{D_6}\overline{M_d}\bar{D_2}\bar{D_1}D_0 \\
I_{ALUopR}	&=& D_7\bar{D_6}\overline{M_s} \\
I_{ALUopM}	&=& D_7\bar{D_6}M_s \\
I_{ALUopI}	&=& \bar{D_7}\bar{D_6}D_2\bar{D_1}\bar{D_0} \\
I_{ROT}		&=& \bar{D_7}\bar{D_6}\bar{D_2}D_1\bar{D_0} \\
I_{JMP}		&=& \bar{D_7}D_6 D_2\bar{D_1}\bar{D_0} \\
I_{JMPc}	&=& \bar{D_7}D_6 \bar{D_2}\bar{D_1}\bar{D_0} \\
I_{CAL}		&=& \bar{D_7}D_6 D_2 D_1\bar{D_0} \\
I_{CALc}	&=& \bar{D_7}D_6 \bar{D_2}D_1\bar{D_0} \\
I_{RET}		&=& \bar{D_7}\bar{D_6}D_2 D_1 D_0 \\
I_{RETc}	&=& \bar{D_7}\bar{D_6}\bar{D_2}D_1 D_0 \\
I_{RST}		&=& \bar{D_7}\bar{D_6}D_2\bar{D_1}D_0 \\
I_{INP}		&=& \bar{D_7}D_6\bar{D_5}\bar{D_4}D_0 \\
I_{OUT}		&=& \bar{D_7}D_6(D_5+D_4)D_0 \\
I_{HLT}		&=& \bar{D_7}\bar{D_6}\bar{D_5}\bar{D_4}\bar{D_3}\bar{D_2}\bar{D_1}+D_7 D_6 M_d M_s
\end{eqnarray}

\subsection{IF转出设计}
IF转出到微地址01000-01111, 转出编码
\begin{eqnarray*}
F_0			&=& I_{HLT} \\
F_1			&=& I_{Lrr}+I_{ALUopR}+I_{LMr} = D_7\bar{M_s} \\
F_2			&=& I_{JMP}+I_{JMPc}+I_{CAL}+I_{CALc}+I_{LrI}+I_{LMI}+I_{ALUopI} = \bar{D_7}\bar{D_0}(D_6+\bar{D_6}D_2) \\
F_3			&=& I_{LrM}+I_{ALUopM} = D_7 M_s(D_6\bar{M_d}+\bar{D_6}) \\
F_4			&=& I_{INP}+I_{OUT} = D_7\bar{D_6}D_0 \\
F_5			&=& I_{INr}+I_{DCr}+I_{ROT} = \bar{D_7}\bar{D_6}\bar{D_2}(\bar{M_d}\bar{D_1}+D_1\bar{D_0}) \\
F_6			&=& I_{RET}+I_{RETc} = \bar{D_7}\bar{D_6}D_1 D_0 \\
F_7			&=& I_{RST} = \bar{D_7}\bar{D_6}D_2\bar{D_1}D_0
\end{eqnarray*}
使用8-3编码器
\begin{eqnarray*}
Y_0^0		&=& F_1+F_3+F_5+F_7 \\
Y_1^0		&=& F_2+F_3+F_6+F_7 \\
Y_2^0		&=& F_4+F_5+F_6+F_7 
\end{eqnarray*}
微地址转换公式如下
\begin{eqnarray*}
A_4 &=& \mu A_4 \\
A_3 &=& \mu A_3 \\
A_2 &=& P_0(Y_2^0)+\mu A_2 \\
A_1 &=& P_0(Y_1^0)+\mu A_1 \\
A_0 &=& P_0(Y_0^0)+\mu A_0
\end{eqnarray*}

\subsection{1级转入设计}
微地址编码
\begin{itemize}
  \item 10000: rW
  \item 10001: ALU
  \item 10010: rL
  \item 10011: PCL
  \item 10100: IFb
  \item 10101: rB
  \item 10110: X3
\end{itemize}


\chapter{基本逻辑}
\section{编码器}
\subsection{8-3编码器}
\begin{tabular}{|c|c|c|c|c|c|c|c||c|c|c|}
\hline
$D_{7}$ & $D_{6}$ & $D_{5}$ & $D_{4}$ & $D_{3}$ & $D_{2}$ & $D_{1}$ & $D_{0}$ & $A_{2}$ & $A_{1}$ & $A_{0}$ \\
\hline
0 & 0 & 0 & 0 & 0 & 0 & 0 & 1 & 0 & 0 & 0 \\
0 & 0 & 0 & 0 & 0 & 0 & 1 & 0 & 0 & 0 & 1 \\
0 & 0 & 0 & 0 & 0 & 1 & 0 & 0 & 0 & 1 & 0 \\
0 & 0 & 0 & 0 & 1 & 0 & 0 & 0 & 0 & 1 & 1 \\
0 & 0 & 0 & 1 & 0 & 0 & 0 & 0 & 1 & 0 & 0 \\
0 & 0 & 1 & 0 & 0 & 0 & 0 & 0 & 1 & 0 & 1 \\
0 & 1 & 0 & 0 & 0 & 0 & 0 & 0 & 1 & 1 & 0 \\
1 & 0 & 0 & 0 & 0 & 0 & 0 & 0 & 1 & 1 & 1 \\
\hline
\end{tabular}

\begin{eqnarray}
A_{0} &=& D_{1}+D_{3}+D_{5}+D_{7} \\
A_{1} &=& D_{2}+D_{3}+D_{6}+D_{7} \\
A_{2} &=& D_{4}+D_{5}+D_{6}+D_{7}
\end{eqnarray}

\end{document}
